This chapter describes what I intend to do until in the coming few months.
I will be advancing the project in two ways, I will be generating hardware for it, and I will be continuing to modify the software involved.

\section{Hardware}
\subsection{PCB}
Once the code is functional on the DSP, the next step is to move it off the evaluation board and onto a specially designed PCB.

\section{Algorithm}
\subsection{LMS}
\label{sec:LMS}
Least Mean Squares (LMS) is an alternative method for achieving the cancellation.
Where my current method could perhaps be considered to be a bit of a ``Brute force'' method, LMS is a cleaner, neater method.

LMS is a form of adaptive filter.
It works by reducing the average power of the feedback signal by modifying the co-efficients in a Finite Impulse Response (FIR) filter.
An example implementation is shown below in figure~\ref{fig:lmsfilter}.
In this implementation the LMS algorithm will be used as the block labelled ``Adaptation algorithm''.
This version also aims for a slightly different effect to the method that will be employed in this project.
This project will be feeding the inverse of the noise estimate, -\^{n}(m), to the listener in order to generate the cancelling in their ear, rather than to remove the noise inside the software.

\begin{figure}[H]
	\centering
	\includegraphics[width=\textwidth]{./img/lmsfilter.png}
	\caption{An example setup for the use of an LMS filter in noise cancellation \cite{AdvancedDSPing}}
	\label{fig:lmsfilter}
\end{figure}

\subsection{ICA}
Independent Component Analysis (ICA) is a method for identifying and separating multiple sound sources without prior knowledge of what they are.
