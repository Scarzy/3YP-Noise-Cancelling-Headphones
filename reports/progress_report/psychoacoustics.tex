\section{Psychoacoustics}
The way that humans hear is by no means a simple matter.
The shape of the ear is very important for filtering the sounds that enter the ear, and then the human brain employs many interesting tactics in order to determine various features of the sounds heard.

The delays between the sound being heard at either ear can be used to determine the location of the noise source.
The frequency response of the physical structure of the head also comes into play, allowing a more accurate judgement of the location and type of the sound source.

Most of the sounds heard will not have just passed directly from the source to the ear, but there are likely to be multiple paths taken by the sound waves.
The result of which is that the same sound is heard multiple times with delays of a fraction of a second between them.
Many of these echoed signals are not perceptible, however this does not stop the brain from utilising them.
The brain is constantly working out information such as the size and basic structure of the surroundings from this.
