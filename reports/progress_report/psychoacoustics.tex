\section{Psychoacoustics}
The way that humans hear is by no means a simple matter.
The shape of the ear is very important for filtering the sounds that enter the ear, and then the human brain employs many interesting tactics in order to determine various features of the sounds heard.

The delays between the sound being heard at either ear can be used to determine the location of the noise source.
The frequency response of the physical structure of the head also comes into play, allowing a more accurate judgement of the location and type of the sound source \cite{CogPsychMus}.

Most of the sounds heard will not have just passed directly from the source to the ear, but there are likely to be multiple paths taken by the sound waves.
The result of which is that the same sound is heard multiple times with delays of a fraction of a second between them.
Many of these echoed signals are not perceptible, however this does not stop the brain from utilising them.
The brain is constantly working out information such as the size and basic structure of the surroundings from these echoes \cite{CogPsychMus}.

For an echo to be perceptible to the human brain the delay must be of at least a certain duration, any echoes with a delay less than this will not be consciously noted.
Echoes with the required duration or longer will be heard, and the length of the duration now dictates how significant the echo is.
Clearly the amplitude of the echo also becomes significant, an echo with a noticeable delay and without any amplitude loss will be more apparent to the listener than an echo with a reduction in the amplitude.
The length of duration required for an echo to become noticeable is approximately 10ms \cite{TimeSpaceHearing}, however there are other times which have a notable effect on the hearing and are commonly used in audio mixing.
Some of these times can be seen in Table~\ref{tab:delaytimes}.

\begin{table}[H]
	\begin{tabular}[c]{| l | l |}
		\hline
		Delay Time			& Resulting Impression \\ \hline
		0ms				& Mono source in center \\
		0ms~\textless~x~\textless~10ms	& Sound source appears to be hard-panned \\ & to non-delayed speaker. Delay otherwise unnoticed \\
		10ms~\textless~x~\textless~30ms	& Impression of amplification and added ``liveness'' \\
		15ms~\textless~x~\textless~25ms	& Impression of stereo signal from mono source \\ & can be generated \\
		30ms~\textless~x~\textless~50ms	& Delay becomes noticeable as a separate signal \\
		\textgreater~50ms		& Obvious echo \\
		\hline
	\end{tabular}
	\label{tab:delaytimes}
	\caption{The effect on listener impression of sound source imaging, produced by different delay times imposed on a signal sent to left and right speakers. Based on Table 10.2 by Daniel M. Thompson \cite{UnderstandingAudio}}
\end{table}

This time restriction has an effect on the rate at which the data must be processed.
Each sample must be dealt with and have the corresponding sample sent back out within the 10ms, otherwise the sound will not be cancelled, but instead produce a high frequency ringing which will annoy the user.
