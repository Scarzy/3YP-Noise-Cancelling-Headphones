\newpage
\section{Project Management}
This project was spread over almost an entire year, as such I needed to maintain a suitable amount of management in order to ensure that work did not fall behind.

\subsection{Gannt Chart}
One method I used to keep track of things was to use a Gannt Chart. This is a way of showing what work should be being done when.

\subsection{Weekly Tracker}
While Gannt Charts clearly show what should be being done when, I felt it would help me more to have a shorter term list of things that needed doing, and a way of measuring how well on track I was at the time. As such I created the 'Weekly Tracker'.
The weekly tracker was a sheet containing a list of tasks to work on during the coming week, including a review date on which I would look back and see how well I'd kept up. Each tracking sheet also included a weekly score.

Each task has a task name, also it had space for notes at the beginning and end of the week. Along with this, each task has a rank from 0 to 10 for minimum acceptable completion. This is essentially the minimum percentage of the task that I need to have completed by the end of the week to remain on track. Using a rank below 10 here allows for tasks to cross multiple weeks, with increasing the required rank as progressing through the project.
There is also a rank for task completion. This rank is out of the entire task, not just the amount required for the week allowing me to note if I had progressed beyond where expected for that week. In the event that I progressed beyond the scope required by the overall task, I would be able to rank this as '+'.
At the end of the week, after completing the ranks for task completion, I could then sum up the differences between required and completed ranks to give me a score for the week.
A negative score would denote that I am behind schedule, a positive score denotes that I'm ahead of schedule, and a score of 0 would indicate that I'm spot on schedule.

\subsection{Milestones}
I set a colletion of milestones for me to achieve. At each one I would be able to demonstrate a level of functionality of the system.
