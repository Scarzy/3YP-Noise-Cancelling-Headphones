\section{Hardware}
A DSP is not the only device that would be capable of supporting the processes required. One example of an alternative device is a Field-Programmable Gate Array (FPGA).

Notes:

DSP over FPGA:
\begin{itemize}
\item Larger memory for chip size (RAM would swallow basically entire chip)
\item Greater flexibility while programming (no hardware size limitations)
\item Pinout fixed, on reprogramming FPGA might want to change to make it physically possible, whereas DSP is fixed pinout and only code changes
\end{itemize}

FPGA over DSP:
\begin{itemize}
\item Potentially faster
\item More flexible for speed considerations (can implement multiple functions at once, while DSP can only do single instruction at a time)
\end{itemize}

ASIC:
\begin{itemize}
\item Much faster
\item Much more expensive
\item Once made, that's it, no modifications, it's fixed
\item Further work needed, possibly even as far down as transister level
\item Much easier to make a mistake
\end{itemize}
