\section{Hardware}
Much of the hardware used in this project is not possible to simulate, due to
the nature of the integrated circuits involved.
However, one aspect that can be simulated is the analogue circuitry before the
codec.

\subsection{Signal Conditioning}
Models of the signal conditioning were created in OrCAD in order to check that
the break frequency and roll off were suitable.
The design of this circuitry will be discussed later, in section
\ref{sec:imple:hard:sch} - only the modelling of it will be covered here.
The result of this modelling can be seen in figure \ref{fig:sigcondmodel}.
\\
\\
This graph shows that the signal conditioning amplifier has a break frequency
of 19409Hz.
This is ideal for the function it will be providing, as it is just below the
boundary of human hearing, meaning that all audible frequencies will reach 
the codec for sampling, while frequencies that would produce aliasing are
nullified.

\begin{figure}[H]
	\centering
	\includegraphics[width=\textwidth]{./img/signal_conditioning_sim.pdf}
	\caption{Simulation results of the signal conditioner}
	\label{fig:sigcondmodel}
\end{figure}

\subsection{Summing Amplifier}
A model of the summing (or instrumentation) amplifier was produced to check that it would function
as desired.
As the amplifier being used was not originally conceived for the use to which
it is put to in this project, modelling was even more important to check that
it would function as desired.
Although the produced amplifier would operate on two channels, as these channels
are identical, only one was simulated.
\\
\\
Figures \ref{fig:instramp} and \ref{fig:instrampbeat} show that the summing amplifier will correctly sum the two signals together, with one of the signals experiencing a 180$^{\circ}$ phase shift.

\begin{figure}[H]
	\centering
	\includegraphics[width=\textwidth]{./img/instrumentationamp.pdf}
	\caption{The output from the instrumentation amplifier from two identical sine waves}
	\label{fig:instramp}
\end{figure}

\begin{figure}[H]
	\centering
	\includegraphics[width=\textwidth]{./img/instrumentationamp_dig.pdf}
	\caption{The output from the instrumentation amplifier from two digital signals}
	\label{fig:instrampbeat}
\end{figure}
