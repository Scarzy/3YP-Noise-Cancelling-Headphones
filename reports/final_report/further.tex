\section{Further Work}

If this project were to be continued, there are a few features that could be looked into.
Some of these affect the hardware, others affect the algorithmic side, and can largely be treated independently.
\\
\\
The PCB has a few flaws.
One of these problems is the current inability to program the device.
Development of a programming method would be advantageous, allowing the algorithms to be tested thoroughly.
Another change that would be ideal would be to remove the instrumentation amplifier in the analogue portion of the system.
I would also recommend a change of the DSP used in the project, the C6713 has a few interesting quirks including occasionally reseting if the code acts unexpectedly even if all reset features, such as the watchdog timer, are disabled.
Finding one with a lower power draw would be preferable, as the device is required to be portable, therefore a long battery life is required.
\\
\\
On the algorithmic side of the setup, LMS implementation could be furthered.
Also experimenting with the use of Independent Component Analysis on the device, in order to remove the need for an external sensor to provide the noise input.
This is covered more in appendix \ref{appendix:ica}.
