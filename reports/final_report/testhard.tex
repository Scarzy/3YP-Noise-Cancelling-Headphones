\section{Hardware}

After production, the PCB was tested to check it matched the required design.
Some problems were discovered before and during testing, these are discussed in appendix \ref{ssec:testHWprobs}.

\subsection{Results}
Despite the problems in production, many parts of the PCB could still be tested.
Test conditions used are stated in appendix \ref{ssec:testHWconds}.
This section only covers the basic results from the testing, a more detailed analysis is in appendix \ref{ssec:testHWresults}.

\subsubsection{Power}
Firstly the power circuitry was tested.
The voltages provided by the regulators, and the LED status were checked.
This section of the circuitry functioned correctly.

\subsubsection{DSP}
As discussed in appendix \ref{ssec:testHWprobs} the DSP was not programmable by the end of the project, this limited the testability of the supporting circuitry.
Attempts were made to confirm the JTAG functionality of the DSP, however they were largely unsuccessful.
\subsubsection{Codec}
Unfortunately, without being able to program the DSP it is not possible to configure the codec to interact.
However, there is a second part to that circuit that can be tested; the clock generator.
\\
\\
The clock generator produced the desired output. 

\subsubsection{Analogue}
The frequency response of the signal conditioning amplifier was tested, to check it would provide the correct attenuation.
An acceptable frequency response was observed.
