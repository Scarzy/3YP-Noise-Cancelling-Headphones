\label{appendix:ica}
Independent Component Analysis (ICA) is a method for separating different signals from a mix.
It is a possible solution to the cocktail party problem.

\section{Methodology}

ICA works on a few assumptions.
Firstly that the signals are non-Gaussian, with at most one exception.
This is because Gaussian signals are symmetrical, and do not contain matrix statistics of order higher than second order.
As such, two Gaussian signals are non-separable.
Secondly that all the signals are statistically independent.
Finally that the number of signal sources is less than or equal to the number of sensors.
\\
\\
The aim of any ICA algorithm is to find a de-mixing matrix W which, when multiplied by the input vector from the sensors, produces a vector containing the values emitted by the various sources.
There are various ways to achieve this.
One of these is where the W for the next time frame is dependent on the gradient of the contrast function.
This function, also known as the divergence, determines how different two matrices are to each other.
Taking the previous values of W, and adding on the gradient values attenuated by a adaptation step size, allows a new version of W to be produced, which homes in on an optimal de-mixing matrix separating out the different sources.

\section{Limitations}
ICA has some limitations which must be met.
Some of these limitations would result in this algorithm being unsuitable for this project, though these limitations could potentially be circumvented through knowing the circumstances of use.
\\
\\
Under normal circumstances ICA requires at least as many sensors as signal sources.
Without this there isn't enough statistical variation for the different signals to be separated out.
This project only allows for two inputs, one in each ear, which would result in the use of ICA only working for 2 sound sources in the room, which is unlikely to occur, making the use of ICA pointless.
In order to solve this problem certain information about the location and construction of the headphones can be utilised.
The headphones are positioned inside the ear with the microphones pointing outwards.
The ear is shaped in order to direct sound inside the ear, also the microphones are rather directional resulting in directional microphones capturing sound from all directions inside opposing hemispheres.
On top of this, the users head is located between the two microphones.
The effect of this is that signals produced by a source on one side of the user are received clearly by the microphone on that side, while the signal received by the other microphone has been affected by the frequency response of the head.
\\
\\
Using these effects it may be possible to determine a significant, statistical variation between the signals received at each microphone.
If such a variation can be determined, then the signals should be separable without the increased number of microphones.

\section{Implementation}
In the final project ICA was not implemented.
This was largely due to the unexpectedly long length of time taken in implementing the noise cancellation algorithm.
