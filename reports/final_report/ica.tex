Independent Component Analysis (ICA) is a method for separating different signals from a mix.

\section{Methodology}

\section{Limitations}
ICA has some limitations which must be met.
Some of these limitations would result in this algorithm being unsuitable for this project, though these limitations could potentially be circumvented through knowing the circumstances of use.

Under normal circumstances ICA requires at least as many sensors as signal sources.
Without this there isn't enough statistical variation for the different signals to be separated out.
This project only allows for two inputs, one in each ear, which would result in the use of ICA only working for 2 sound sources in the room, which is unlikely to occur, making the use of ICA pointless.
In order to solve this problem certain information about the location and construction of the headphones can be utilised.
The headphones are positioned inside the ear with the microphones pointing outwards.
The ear is shaped in order to direct sound inside the ear, also the microphones are rather directional resulting in 

\section{Implementation}
In the final project ICA was not implemented.
This was largely due to the unexpected length of time taken in implementing the noise cancellation algorithm.
