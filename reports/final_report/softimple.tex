\section{Software}
The algorithmic side of the system was implemented in the C programming language, due to the amount of control it provided and the fact that the available compilers required C.

\subsubsection{Ring Buffers}

Some data needed to be stored in the DSP for the algorithm to work, whether the correlation method or the LMS method.
There are a few ways to achieve this.
One option is an array.
This would give the advantage of a simple system, whereby the latest value is loaded into the array.
The issue then arises on how to determine the location in the array to load the value, and how to maintain a consistent amount of storage.
If there's no wrapping around at the end of the array, then one of two options are available.
Either the new value is entered at the beginning of the array, and all other values get shifted along.
This method is very heavy on resources as it requires all the values in the arrays to be moved for each new sample.
The alternative is that the amount of data available for each sample varies, dropping down to just a single sample at times during the analysis.
This latter method is basically unacceptable for this project, it would result in the algorithm being unable to respond at portions of the time.
Alternatively the option of wrapping round at the end of the array, this is the first step towards a ring buffer.
Wrapping around at the end of an array removes the need to shift all values along upon receiving a new sample, it also prevents varying available data sizes.
However it then introduces an issue of keeping track of the location in the array and how to deal with the wrap-around.
Multiple variables would have to be stored and whenever a piece of code works with the array it would have to keep track of the wrapping around, introducing a large margin for error in coding.
\\
\\
To eliminate this, a ring buffer struct was developed.
This allowed the required variables to be linked directly with the array being dealt with, and shifting through the array could now be done with a standard function that only required a single argument.
The result of using ring buffers was that the code became simpler to write and deal with, reducing room for error, however this came at a cost.
Using a simple array based method would have resulted in minimal computational effort in shifting through the array, with ring buffers more effort was required, slowing down the overall operation.
Thanks to the speed of the DSP the number of clock cycles this would cost was minimal, and as such ring buffers were used for the remainder of the project.
\\
\\
The use of ring buffers provided other advantages.
After implementing the different algorithms optimisation was required, and the ring buffers assisted with that greatly.
For example, when calculating the mean of the data in the buffer, it is known that the next sample in the buffer, regardless of position in the buffer, is the oldest sample.
Therefore, in order to determine the mean and additional variable could be stored holding the sum of the values in the buffer, and on a new sample the oldest sample can be subtracted, the new value added, and then a single division made.
As a result calculating the mean is reduced to a single addition, single subtraction, single division, and a single shift through the buffer, rather than an addition and shift for each value in the buffer, followed by a division.
This is a huge time saving, and as the number of clock cycles is heavily restricted a saving this large is a real advantage.


\begin{itemize}
\item Ring buffers
	\begin{itemize}
	\item ADV.
	\item Optimisation of sums, like mean
	\item Not needing to worry about location in array, just use ptr
	\item Easy to implement
	\item DIS.
	\item Complexity in implementation
	\item Extra computation required
	\end{itemize}
\item Isolating codec comms from algorithm
	\begin{itemize}
	\item ADV.
	\item Simplicity of changing codec interface
	\item Ease of testing
	\item DIS.
	\item Multiple files
	\end{itemize}
\item Self writing all code
	\begin{itemize}
	\item ADV.
	\item Suited own data structures
	\item Allowed optimisations to suit structures
	\item DIS.
	\item Everything had to be written
	\item Lack of some optimisations
	\end{itemize}
\end{itemize}

\subsection{Correlation}

After writing the code for the correlation algorithm, a problem was discovered.
The algorithm for cross-correlation requires the use of the mean in the calculation.
Whilst the mean could be calculated easily due to the 

\subsection{LMS}

