\section{Software}
Many decisions were made in the implementation of the algorithms in this project.
Some of these decisions affected just the correlation method, some just the LMS method, and some that affected the project regardless of methodology.
Here these decisions are discussed, starting with the decisions that were method independent, then covering each of the method specific ones in turn.

\subsubsection{Language}

The algorithmic side of the system was implemented in the C programming language, due to the amount of control it provided and the fact that the available compilers required C.
This gave a large degree of control to the various peripherals built into the DSP, specifically a nice interface to the McBSP registers for communication with the codec, and the GPIO port for using the debugging LEDs.
The disadvantage this introduced was that built in data types and the available functions that could be acted upon them were a bit more restrictive than those used in the modelling with Matlab.
As such the desired structs and functionality for the ring buffers (discussed below) had to be designed and developed manually.
This also affected the algorithms themselves, as each value had to be dealt with individually, whereas Matlab could perform matrix based operations.

\subsubsection{Ring Buffers}

Some data needed to be stored in the DSP for the algorithm to work, whether the correlation method or the LMS method.
There are a few ways to achieve this.
One option is an array.
This would give the advantage of a simple system, whereby the latest value is loaded into the array.
The issue then arises on how to determine the location in the array to load the value, and how to maintain a consistent amount of storage.
If there's no wrapping around at the end of the array, then one of two options are available.
Either the new value is entered at the beginning of the array, and all other values get shifted along.
This method is very heavy on resources as it requires all the values in the arrays to be moved for each new sample.
The alternative is that the amount of data available for each sample varies, dropping down to just a single sample at times during the analysis.
This latter method is basically unacceptable for this project, it would result in the algorithm being unable to respond at portions of the time.
Alternatively there is the option of wrapping round at the end of the array, this is the first step towards a ring buffer.
Wrapping around at the end of an array removes the need to shift all values along upon receiving a new sample, it also prevents varying available data sizes.
However it then introduces an issue of keeping track of the location in the array and how to deal with the wrap-around.
Multiple variables would have to be stored and whenever a piece of code works with the array it would have to keep track of the wrapping around, introducing a large margin for error in coding.
\\
\\
To eliminate this, a ring buffer struct was developed.
This allowed the required variables to be linked directly with the array being dealt with, and shifting through the array could now be done with a standard function that only required a single argument.
The result of using ring buffers was that the code became simpler to write and deal with, reducing room for error, however this came at a cost.
Using a simple array based method would have resulted in minimal computational effort in shifting through the array, with ring buffers more effort was required, slowing down the overall operation.
Thanks to the speed of the DSP the number of clock cycles this would cost was minimal, and as such ring buffers were used for the remainder of the project.
\\
\\
The use of ring buffers provided other advantages.
After implementing the different algorithms optimisation was required, and the ring buffers assisted with that greatly.
For example, when calculating the mean of the data in the buffer, it is known that the next sample in the buffer, regardless of position in the buffer, is the oldest sample.
Therefore, in order to determine the mean an additional variable could be stored holding the sum of the values in the buffer, and on a new sample the oldest sample can be subtracted, the new value added, and then a single division made.
As a result calculating the mean is reduced to a single addition, single subtraction, single division, and a single shift through the buffer, rather than an addition and shift for each value in the buffer, followed by a division.
This is a huge time saver, and as the number of clock cycles is heavily restricted a saving this large is a real advantage.

\subsubsection{Code Isolation}
In order to ease the writing and testing of the code, separate parts of the code were written in separate files, with minimal requirements from other files.
All the code that interfaced with external devices, such as the code relating to the codec communications, was especially separated out, with only functions
to get and send data used externally.
This gave the advantage that all the algorithmic code was cross platform, enabling it to be tested on normal computer systems.
Of course to test it on a normal computer system, the I/O calls and the codec interface were no longer valid, so had to be mimicked.
This mimicking was achieved through a minimalistic block of code which simply read from a file and wrote to the console.
\\
\\
This ability to test it cross platform proved especially useful when a ring buffer pointer was not being shifted through the buffer correctly, resulting in a memory leakage.
The DSP just followed its programming and continued, overflowing and overwriting program memory.
This resulted in unexpected crashes on both the DSP and its debugger, which just looked like the code had frozen.
When tested in a Linux terminal, the memory overflow was detected and indicated as a segmentation fault with a complete stack trace, allowing the bug to be very easily traced.

\subsubsection{Self Written}
In this project none of the C code used was taken from an available library or provided by anyone not involved in this project.
Some of the Matlab code written during the modelling used inbuilt functions to begin with, in order to produce proof of concept, however then was re-written to remove these external functions.
\\
\\
There were multiple advantages to writing all the code.
One of these was the advantage of using the desirable data structures, in particular the ring buffers.
While many libraries available deal with ranges of data, they generally expect arrays and do not provide the capability to deal with the ring buffers used in the rest of the project.
This would result in having to resort to the alternatives, resulting in serious losses to functionality.
\\
\\
Of course, no advantages come for free, and as such there were multiple disadvantages to this approach.
The most obvious one being that all the code had to be written and tested, whereas this would have been done extensively with any available libraries.
On top of this, libraries are available for the DSP which achieve a variety of functions, and have had the assembly code hand optimised by the manufacturer, Texas Instruments (TI).
These optimisations were lost, resulting in slightly slower code.
However many of these libraries were incompatible with the data structures used by the rest of the system, so the losses in optimisations here were made up for through optimisation in other ways, as previously discussed.

\subsection{Correlation}
The correlation code was implemented in two phases.
Initially the code was written in a bloated manner, where the code was written to be ultimately functional and match the algorithm rather than optimal.
Once this code was achieved and tested for functionality it was rewritten to be more optimal.
This re-writing was based around merging loops through the data that were capable of being run side by side.
\\
\\
The cross correlation is determined of the two signals, and then this value is stored in an array.
The peak of this array was then detected.
An array was used here rather than a ring buffer as all the values would change on each new sample, as all the values are dependant on the mean.
Therefore the advantages of a ring buffer are nullified.

\subsection{LMS}
\label{ssec:implelms}
The LMS code acted in three stages; calculate output, generate feedback, and update weights.
The first of these stages, calculating the output, involved iterating over a loop summing up the various tapped values.
The weights were stored in an array, and matched against the values in the ring buffer by producing a new pointer for it, and shifting it round backwards.
After multiplying respective weights and stored values, the output of this filter is then played out through the headphones.
\\
\\
The seconds stage involved the generation of a virtual feedback value to be used for the final stage.
The generation of this assumed that the sounds reaching the external microphone of the headphones were the same as those reaching the eardrum.
This assumption was slightly dangerous as it did not account for the demanded sound signal, which would be reaching the eardrum also.
\\
\\
The final stage involved updating the array of weights to optimise the cancellation, based on the feedback.
The algorithm for calculating the updated weights works on the basis of minimising the power of the virtual feedback signal.
At this point the assumption made in the second stage is validated, as the demanded signal would just increase the minimum power level.
Once the updated weights have been calculated and stored in the array the loop repeats.
