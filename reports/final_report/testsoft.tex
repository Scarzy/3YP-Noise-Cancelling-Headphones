\section{Software}

Although it was not possible to test the algorithmic on the hardware produced for this project, due to the aforementioned programming issues, it was possible to test the algorithms on a development board using the same DSP as used on the PCB.

\subsection{Correlation}
After writing the code for the correlation algorithm, a problem was discovered.
The algorithm for cross-correlation requires the use of the mean in the calculation.
Whilst the mean could be calculated easily due to the advantages provided through the use of ring buffers, there were additional issues with inserting this mean at the correct point in the calculation.
The entire buffer had to be iterated through after the calculation of the mean, with a calculation performed at each step.
This second calculation was more intensive than the calculation of the mean would have been.
This produced a severe draw on the processing power of the DSP, and resulted in the system taking approximately 7 seconds between each sample.
As the codec operated at 48KHz, this meant that the code was operating about \emph{three hundred thousand} times too slow.
\\
\\
Two options presented themselves in order to reduce this issue.
Firstly the length of the buffers could be reduced.
This would have been a simple fix, and would have drastically speeded up the algorithm.
However, reducing the buffer size by enough to produce a suitably fast algorithm would have removed so much information than would have been desirable for accurate cancelling.
Another effect of shortening the buffers would be that the headphones could no longer deal with as long a difference between the heard signal and the noise signal.
As a result, the cancelling effect would be reduced.
\\
\\
The other option would be to optimise the code by as many means possible, by looking for loopholes in how the algorithm iterates through the loops.
\\
\\
To a certain degree both methods were used to speed up the system, however were not effective enough.

\subsection{LMS}

The LMS algorithm was tested in two ways.
Firstly the code was tested through a Linux terminal.
When given two sine waves inputs the algorithm significantly attenuated the signal.
Secondly the code was tested on the development board.
Further details about these tests are discussed in appendix \ref{ssec:testSWlms}.
