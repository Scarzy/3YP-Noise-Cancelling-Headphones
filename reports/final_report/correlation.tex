\section{Correlation}

The first of the two methods tried was based on correlation.
Cross correlation is a way of determining how similar two signals are, and how much of a delay there is between the two.
\\
\\
In order to achieve this, on each sample the heard signal and the noise signal were cross correlated.
If the heard signal contains the noise signal then this appears as a peak in the cross correlation.
This peak can then be detected, along with its position.
The position allows for a shift to be accounted for.
\\
\\
Although the heard sound may well contain the noise sound, there is no guarantee that the noise sound will not have undergone some form of distortion; as a matter of fact it is very likely that it will have been distorted.
The two most likely effects for the noise to suffer are delay, and a decrease in amplitude.
A delay is caused when the user is at a different distance from the noise source than the noise input is.
As such the noise reaches one a few samples before reaching the other.
This can also result in a phase shift.
In order to account for this the correlation determines the position of the peak, this corresponds to the delay between the two detections.
This shift is then applied to the noise signal, to bring it in line with its component in the heard signal.
The inverse of the last detected sample is then added to the output, in order to cancel with its corresponding component reaching the users ears.
